\documentclass[a4paper,10pt]{article}
\usepackage[utf8x]{inputenc}
\usepackage[bulgarian]{babel}
\usepackage{anttor}
%\usepackage[OT2,T2A]{fontenc}
\usepackage[OT2]{fontenc}
\usepackage{array}
\usepackage{graphicx}
\usepackage{lscape}
\usepackage{lastpage}
\usepackage{amssymb}
\usepackage[body={6.5in,8.85in}, top=0.2in, right=1.0in, bottom=0.2in ]{geometry}
\pagestyle{empty}

\begin{document}

\textbf{Зад. 1.} 
При аварийно спиране, автомобил движещ се със скорост 72 km/h спира за 5 s.
Намерете спирачния път. Ускорението на автомобилът да се приеме за постоянно.

\textbf{Зад. 2.} 
Спускайки се скиор изминава 100~m за 20~s, движейки се с ускорение 0.3~$\mathrm{m/s^2}$.
Каква е скоростта на скиорът в началото и края на пътят?


\textbf{Зад. 3.} 
Колко пъти скоростта на куршум в средата на цевта на ловджийска пушка намалява при излизане от цевта?
Движението на куршумът в цевта считаме за равноускорително.


\textbf{Зад. 4.}
Двете материални точки се движат по оста х равномерно със скорост $v_1$~=~8~ m/s и $v_2$~=~4~m/s. 
В първоначалния момент, първата точка е от лявата страна на координатната система на разстояние 21 м,
а втората - в дясно на разстояние 7 метра.
След колко време първата точка ще догони втората? Къде ще се случи това?
Начертайте графиката на движението.

\textbf{Зад. 5.}
Разстоянието между две точки в началният момент е 300 m.
Точките се движат една към друг със скорости 1.5 m/s и 3.5 m/s.
Когато се срещат? Къде ще се случи това?
Начертайте графиката на движението.

\textbf{Зад. 6.}
Катер се движи по течението от т. А до т. Б за време $t_1$ = 5h.
Какво време изразходва катера при обратния път, ако
скоростта на катера относно водата е n=5 пъти скоростта на течението?
\begin{flushright}
Отг. $t=\frac{n+1}{n-1}t_1$ 
\end{flushright}

\textbf{Зад. 7.}
От град А за град Б по прав път тръгва товарен камион със скорост $v_1$ = 40 km/h.
След време $t_0$= 1.5 h от Б към А тръгва автомобил със скорост $v_2$ = 80 km/h.
След колко време t от тръгването на автомобилът и на какво разстояние d от Б
се срещат камионът и автомобилът, ако в момента на пристигане на автомобилът в А,
камионът е изминал път s=100 km.
\begin{flushright}
Отг. $t=\frac{SV_2}{V_1(V_1+V_2)} - t_0,\qquad d=tv_2$ 
\end{flushright}

\textbf{Зад. 7.}
Тяло е хвърлено нагоре от височина 20~m с начална скорост 3~m/s.
На каква височина ще се намира тялото 2~s  след началото на движението?
\begin{flushright}
Отг. 6 m 
\end{flushright}

\textbf{Зад. 8.}
В последната секунда на свободно падане на тялото с нулева начална скорост, тялото е изминало  двойно повече път, отколкото през предходната секунди.
От каква височина е паднало тялото.
\begin{flushright}
Отг. 31.25 m 
\end{flushright}

% \textbf{Зад. 9.}
% Тяло е хвърлено вертикално нагоре. 
% Интервалът от време между два момента, когато тялото минава през точка, намираща се на височина H, е $\mathrm{t_0}$.
% Намерете началната скорост.
% 
% 
% \textbf{Зад. 10.}
% Реактивен самолет лети със скорост $v_0$. 
% В даден момент от време самолета започва да се движи с постоянно ускорение в течение на време $t_0$ и в последната секунда минава път S. 
% Определете ускорението и крайната скорост на самолета.
% 
% 
% \textbf{Зад. 11.}
% Тяло се движи праволинейно от точка А и се движи в началото равноускорително в течение на време $t_0$, след това с модул на ускорение - равнозакъснително. 
% След колко време от началото на движението тялото ще се върне в т. А?



\end{document}

% Велосипедист се движи между 2 града.
% Половината път изминава със скорост $v_1$ =12 kh/h.
% Другата половина ..
